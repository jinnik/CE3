\documentclass{article}
\usepackage{amsmath}
\usepackage{pgfplots}
\usepackage{pgfplotstable}
% \pgfplotsset{compat=1.14}
\usepgfplotslibrary{groupplots}
\usetikzlibrary{spy}
% \usetikzlibrary{external} % Probaby doesn't work on overleaf :(
\usepackage{listings}
\usepackage{siunitx}
\usepackage{cleveref}
\usepackage{booktabs}
\usepackage{color} %red, green, blue, yellow, cyan, magenta, black, white
\definecolor{mygreen}{RGB}{28,172,0} % color values Red, Green, Blue
\definecolor{mylilas}{RGB}{170,55,241}

\lstset{language=Matlab,%
    %basicstyle=\color{red},
    breaklines=true,%
    morekeywords={matlab2tikz},
    keywordstyle=\color{blue},%
    morekeywords=[2]{1}, keywordstyle=[2]{\color{black}},
    identifierstyle=\color{black},%
    stringstyle=\color{mylilas},
    commentstyle=\color{mygreen},%
    showstringspaces=false,%without this there will be a symbol in the places where there is a space
    numbers=left,%
    numberstyle={\tiny \color{black}},% size of the numbers
    numbersep=9pt, % this defines how far the numbers are from the text
    emph=[1]{for,end,break},emphstyle=[1]\color{red}, %some words to emphasise
    %emph=[2]{word1,word2}, emphstyle=[2]{style},    
}

\pgfplotsset{/pgfplots/table/tableformat/.style={
  col sep=semicolon,  % the separator in our .csv file
  every head row/.style={before row=\toprule,after row=\midrule},
  every last row/.style={after row=\bottomrule},
  every column/.append style={numeric type, dec sep align},
  },
}

\def\d{\mathrm{d}}
% \tikzexternalize

\date{2018-09-14}
\author{Jing Ying Ko jyko@kth.se 
\and Henrik Grimler hgrimler@kth.se}
\title{Computer exercise 3 in SF2520 \\Partial differential equation of parabolic type \\CE3 37 }
\begin{document}

\maketitle
\section{Part 1: Rescaling to dimensionless form}

The following partial differential equation is given:
\begin{equation} \label{eq1}
\rho c_p \frac{\partial T}{\partial t} = \kappa \frac{\partial^2 T}{\partial x^2}
\end{equation}
with boundary conditions
\begin{equation} \label{eq1}
   T(0,t)= 
\begin{cases}
    T_0,           & 0 \leq t \leq t_P, \\
    0,              & t > t_P,
\end{cases}
   \hspace{0.8cm}         \frac{\partial T}{\partial x}(L,t)=0,
\end{equation}
and initial conditions
\begin{equation} \label{eq1}
   T(x,0)= 
\begin{cases}
    T_0,           & x=0, \\
    0,              & 0 < x \leq L
\end{cases}
\end{equation}
where $\rho$ is the density [$kg/m^3$], $c_p$ is the heat capacity [$J/kg \cdot C$] and $\kappa$ is the thermal conductivity [$J/m\cdot s \cdot C$] of the rod.
\newline

\noindent
1. Introduce reference values ($T_0$,$L$,$t_P$) for each of the three variables ($T,x,t$) and define the corresponding dimensionless variables
\begin{equation}
    T = T_0 u, \hspace{0.8cm} x = L \xi, \hspace{0.8cm} t = t_P \tau.
\end{equation}

\noindent
2. Rewrite differential equations in those variables
\begin{equation*}
    \frac{\partial T}{\partial t} = \frac{T_0}{t_P} \frac{\partial u}{\partial \tau}
\end{equation*}
\begin{equation*}
    \frac{\partial}{\partial x} = \frac{1}{L} \frac{\partial}{\partial \xi}
\end{equation*}
\begin{equation*}
    \frac{\partial^2 T}{\partial x^2} = \frac{T_0}{L^2} \frac{\partial^2 u}{\partial \xi^2}
\end{equation*}
Rewrite Equation~\ref{eq1} into
\begin{equation*}
    \rho c_p \frac{\partial u}{\partial \tau} = \kappa \frac{T_0}{L^2} \frac{\partial^2 u}{\partial \xi^2}
\end{equation*}
\begin{equation*}
    \frac{\partial u}{\partial \tau} = \frac{\kappa t_p}{\rho c_p L^2} \frac{\partial^2 u}{\partial \xi^2}
\end{equation*}
and therefore can be transformed into the following dimensionless form
\begin{equation*}
    \frac{\partial u}{\partial \tau} = a \frac{\partial^2 u}{\partial \xi^2}, \hspace{0.8cm} \tau > 0, \hspace{0.8cm} 0 < \xi < \frac{L}{L}=1
\end{equation*}
with boundary conditions
\begin{equation*} 
   u(0,\tau)= 
\begin{cases}
    \frac{T_0}{T_0} = 1,           & 0 \leq \tau \leq \frac{t_P}{t_P}, \\
    0,              & \tau > 1,
\end{cases}
   \hspace{0.8cm}         \frac{\partial u}{\partial \xi}(1,\tau) = 0,
\end{equation*}
and initial conditions
\begin{equation*} 
   u(\xi,0) = 
\begin{cases}
    \frac{T_0}{T_0} = 1,           & \xi = 0, \\
    0,              & 0 < \xi \leq \frac{L}{L}=1
\end{cases}
\end{equation*}


\end{document}
